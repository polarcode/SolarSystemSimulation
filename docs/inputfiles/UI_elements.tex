\chapter{UI Elements}

The modle provides three UI elements for the user, besides the common controls from {\em MatLab} itself. A display for the number of frames rendered pre second, a slider to control the speed at which the modle runs and a legend for the different orbits of the planets and the moon.

\section{FPS Display}
While programming the modle we soon realized that the certain changes had a massive impact on the number of frames {\em MatLab} could render per second. The number of frames per second, also known as {\em FPS} directly dictates how smooth the modle runs. Through time analysis of the modle we found out that the resolution of the planets, sun, and moon had the most noticeable impact on the {\em FPS.} We then added the display to find the optimal resolution at which the graphics of the planets didn't suffer to much but still allowed for a smooth animation. We settled for a resolution of 50. This keeps the {\em FPS} at around 20 which is only slightly lower than the number images the human eye can capture per second. This makes the animation appear smooth.\\

\textcolor{red}{CALCULATION OF FPS}
\begin{framed}\begin{verbatim}
Some code
\end{verbatim}\end{framed}


\section{Speed Slider}
Because Uranus and Neptune are so far away from the sun, it takes them quite a while to complete an entire turn around the sun. That is why we added a slider on the bottom left which allows users to speed up the animation. Additionally the slider can also stop the animation by setting the speed to zero which allows for closer inspection of the planets. The slider basically controls the angle of rotation per tick. Faster speed results in a grater angle. Setting the speed to higher values leads to visual effects like planets appearing to jump rather than smoothly moving and planets smoothly to be turning backwards. Alternatives like controlling a delay for rendering the next frame have been tested, but didn't work out due to the {\em FPS} plummeting or the animation quickly reaching a maximum speed and therefore not having the desired effect.

\section{Orbit Legend}
The third UI element we add is a simple legend for the orbits: it associates a name to the orbit color used. This is especially useful when viewing the entire modle.